\documentclass[uplatex,dvipdfmx]{jsarticle}
\usepackage{amsmath,amssymb,amsthm}
\usepackage{mathtools}
\usepackage{mathrsfs}
% \usepackage[top=2.5cm, bottom=2.5cm]{geometry}
\usepackage[dvipdfmx]{graphicx,hyperref}
\usepackage{here}
\usepackage{enumerate}
\usepackage{algorithm}
\usepackage{algpseudocode}
\usepackage{colortbl}
\usepackage{color}
\usepackage{listings}
\usepackage{tikz}

\DeclareMathOperator*{\argmin}{arg\,min}
\DeclareMathOperator*{\argmax}{arg\,max}
\DeclareMathOperator*{\rank}{rank}

\newcommand{\transposed}[1]{#1^\mathsf{T}}

\newtheorem{lemma}{Lemma}
\newtheorem{definition}{Definition}

\title{数値解析レポート 12/21出題分}
\author{J4-190507 木下裕太}

\begin{document}
  \maketitle

  選択した問題: (1), (2)

  \section*{(1)}
  以下では断りのない限り実数の範囲で考える.

区間$(a,b)$, 重み$w(x)$上の直交多項式系を$\{P_n(x)\}$とする.
初めに次の補題を示す.

\begin{lemma}
  任意の非負整数$n$に対し, $P_n(x)$は相異なる$n$個の実根を持つ.
\end{lemma}

\begin{proof}
  多項式$Q(x),R(x)$が
  \begin{equation*}
    P_n(x)=Q(x)R(x)
  \end{equation*}
  を満たし, $Q(x)$は定数でないとする. $R(x)$は高々$n-1$次であるから
  \begin{eqnarray*}
    0=\int_a^b P_n(x)R(x) w(x)dx=\int_a^b Q(x)R(x)^2 w(x)dx
  \end{eqnarray*}
  $R(x)^2w(x)\neq 0$は常に非負値をとるので, $Q(x)$は区間$(a,b)$上で正値, 負値のいずれもとる.\\
  この事実により, $P_n(x)$の根は$n$個の実根を持ち, かつそれらは相異なることが分かる.

  % 次に実数$\alpha$, 多項式$S(x)$が
  % \begin{eqnarray*}
  %   P_n(x) = (x-\alpha)^2 S(x)
  % \end{eqnarray*}
  % を満たすとする. $S(x)$は高々$n-2$次であるが
  % \begin{eqnarray*}
  %   0=\int_a^b P_n(x)S(x) w(x)dx=\int_a^b (x-\alpha)^2S(x)^2 w(x)dx>0
  % \end{eqnarray*}
  % となり矛盾を得る. ゆえに$P_n(x)$は重根を持たない.
\end{proof}

上の補題で存在が示された, $P_n(x)$の相異なる実根を$\{x_k\}_{k=1}^n$とし,各$k\in\{1,\cdots, n\}$について$n-1$次多項式を
\begin{eqnarray*}
  l_k(x)=\prod_{i\neq k} \cfrac{x-x_i}{x_k-x_i}
\end{eqnarray*}
と定める.
また$f(x)$を高々$2n-1$次の任意の多項式とする.
このとき, 高々$n-1$次の多項式
\begin{eqnarray*}
  g(x)=\sum_{k=1}^n f(x_k)l_k(x)
\end{eqnarray*}
は$P_n(x)$の$n$個の零点全てで$f(x)$と一致する.
したがって$g(x)$は$f(x)$を$P_n(x)$で割った余りであり,
\begin{equation*}
f(x)=g(x)+P_n(x)h(x)
\end{equation*}
なる高々$n-1$次の多項式$h(x)$が存在する. よって
\begin{eqnarray*}
  \int_a^b f(x)w(x)dx &=& \int_a^b g(x)w(x)dx + \int_a^b P_n(x)h(x) w(x)dx\\
  &=& \sum_{k=1}^n f(x_k)\int_a^b l_k(x)w(x)dx
\end{eqnarray*}
であり, 示すべき主張が得られた.

  \section*{(2)}
  便宜上$t_i = (m + \frac{i}{2})\Delta t $と定める.
  考えている範囲で$F$のあらゆる偏導関数は有界.

  $y$の$t_1$を中心としたTaylor展開を考えると
  \begin{eqnarray*}
    y(t_2)-y(t_0)=y'(t_1)\Delta t + O(\Delta t^3)\\
    y(t_2)+y(t_0)=2y(t_1) + O(\Delta t^2)
  \end{eqnarray*}
であり, $F$についても$(t_1, t_1)$を中心としたTaylor展開を考えると
\begin{equation*}
  F(y(t_2), y(t_0))=F(y(t_1), y(t_1))
  +(y(t_2)-y(t_1))\cfrac{\partial F}{\partial x_1}(t_1,t_1)
  +(y(t_0)-y(t_1))\cfrac{\partial F}{\partial x_2}(t_1,t_1)
  +O(\Delta t^2)
\end{equation*}
となる. 与えられた$F$の性質から
\begin{eqnarray*}
  y'(t_1) = F(y(t_1), y(t_1))\\
  \cfrac{\partial F}{\partial x_1}(t_1,t_1) = \cfrac{\partial F}{\partial x_2}(t_1,t_1)
\end{eqnarray*}
が成り立つことをふまえると,
\begin{eqnarray*}
R = (y(t_2)+y(t_0)-2y(t_1))\cfrac{\partial F}{\partial x_1}(t_1,t_1)+O(\Delta t^2)
=O(\Delta t^2)
\end{eqnarray*}
という評価を得る.

\end{document}
