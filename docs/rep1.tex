\documentclass[uplatex,dvipdfmx]{jsarticle}
\usepackage{amsmath,amssymb,amsthm}
\usepackage{mathtools}
\usepackage{mathrsfs}
% \usepackage[top=2.5cm, bottom=2.5cm]{geometry}
\usepackage[dvipdfmx]{graphicx,hyperref}
\usepackage{here}
\usepackage{enumerate}
\usepackage{algorithm}
\usepackage{algpseudocode}
\usepackage{colortbl}
\usepackage{color}
\usepackage{listings}
\usepackage{tikz}

\DeclareMathOperator*{\argmin}{arg\,min}
\DeclareMathOperator*{\argmax}{arg\,max}
\DeclareMathOperator*{\rank}{rank}

\newcommand{\transposed}[1]{#1^\mathsf{T}}

\newtheorem{lemma}{Lemma}
\newtheorem{definition}{Definition}

\title{数値解析レポート 11/2出題分}
\author{J4-190507 木下裕太}

\begin{document}
\maketitle

\section*{(2)}
修正Gram-Schmidt法では正規直交基底を構成するベクトル$q$が1つ求まるたびに、
まだ直交化されていないベクトル全てを${\rm span}(q)$の直交補空間に射影している。
したがって$j$ステップ後では通常のGram-Schmidt法における直交化を第$j+1$項までで打ち切った値が格納されている。

\section*{(3)}
$\rho(H)<1$ならば、与えられた事実から、ある$\mathbb{R}^n$上のノルム$\|\cdot\|$及び定数$\delta\in[0,1)$が存在して、それが誘導する行列ノルムに関して$\|H\|\le\delta$。
このとき
\begin{eqnarray*}
  \forall x,y\in\mathbb{R}^n,\ \|(Hx+c)-(Hy+c)\|\le\|H\|\|x-y\|\le\delta\|x-y\|
\end{eqnarray*}
が成り立つので、縮小写像の原理から主張が従う。

\section*{(4)}
\begin{verbatim}
  import numpy as np    # numpy のインポート

  exa = np.pi ** 2 / 6.0  # 正解の級数和

  prev = np.float32(0.0)  # 単精度として定義
  now = np.float32(0.0)   # 単精度

  for k in range(5000, 0, -1):
      now = now + np.float32(1.0/k**2)   # 級数の一項を足す

  print(exa, now)        # 真値と計算値を表示
\end{verbatim}

\end{document}
